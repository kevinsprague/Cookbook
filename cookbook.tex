\documentclass[12pt]{cookbook}
\title{My Recipes}
\usepackage{url}
\author{Kevin Sprague}
\date{\today}
\begin{document}

% begin{recipe} to begin a recipe that breaks page afterwards
% begin{recipe}[multi] to begin a recipe that doesn't break page
\begin{recipe}{Lilac Simple Syrup}[multi]

\blurb{If you don't add the purple food coloring, it will look yellow/brown, but when added to something like lemonade, it will turn the lemonade pink/purple.}
\yield{2 cups simple syrup}
\acttime{20-30 minutes}
\inacttime{3-8 hours}
\source{Spruce Eats}
\ingredients
	{4c Lilac Flowers}
	{2c Water}
	{2c Sugar}
	{\textbf{Optional:} a couple of drops of purple food coloring}`
\stopingredients
\preparation
	{Rinse the lilac blossoms under cold water to clean off dust/pollen/bugs.}
	{Add the water and sugar to a medium saucepan and heat on medium until the sugar has dissolved and the syrup is clear.}
	{Reduce heat to low and add the lilacs. Cover the pan and heat for five minutes.}
	{Remove from heat and let steep for at least 3 hours or up to overnight.}
	{Strain and squeeze as much liquid from the flowers as possible.}
	{Add the food coloring if using.}
	{Place in airtight container and refrigerate for up to a month.}
\stopprep
\end{recipe}
\begin{recipe}{Lilac Cocktail}
\source{\url{https://emilyfabulous.com}}
\ingredients
    {2 oz gin}
    {1 oz lemon juice}
    {1 oz lilac syrup}
    {1 egg white}
\stopingredients
\preparation
    {Add the gin, lemon juice, lilac syrup, and egg white to a cocktail shaker. Dry shake (no ice) to create a foam}
    {Add fresh ice and shake again.}
    {Strain into a cocktail glass.}
\stopprep
\end{recipe}
\begin{recipe}{Hot Chocolate}
\source{Cooking Light}
\begin{center}
\acttime{15-20 minutes}
\yield{4 mugs of hot cocoa}
\end{center}
\ingredients
    {\sfrac{2}{3} c boiling water}
    {2 oz dark/bittersweet (60-70\% cocoa) chocolate, finely chopped}
    {1 \sfrac{1}{3} c milk}
    {1 cup espresso or strong coffee}
    {\sfrac{1}{4} c cocoa powder}
    {\sfrac{1}{4} c packed brown sugar}
    {1 2-inch strip orange rind}
    {\textbf{Optional:} whipped cream}
\stopingredients
\preparation
    {Combine the boiling water and chopped chocolate in a medium saucepan, stirring until the chocolate melts.}
    {Add milk, coffee, cocoa powder, brown sugar, and orange rind. Cook over medium-low heat, stirring with a whisk, for five minutes or until tiny bubbles form around the edge of the pan, stirring frequently (do not boil)}
    {Discard the orange rind and pour into 4 mugs.}
    {Top with whipped cream if desired.}
\stopprep
\end{recipe}
\begin{recipe}{Black Walnut Social}
\source{\url{https://emilyfabulous.com}}
\ingredients
{2 oz gin (barrel aged or Old Tom if possible)}
{1 oz cardamaro}
{3-4 dashes black walnut bitters}
{\sfrac{3}{4} oz lemon juice}
{orange peel for garnish}
\stopingredients
\preparation
{Place a large ice cube into a rocks glass.}
{Add the gin, cardamaro, bitters, and lemon juice to a mixing glass with ice and stir.}
{Strain into the rocks glass, garnish with the orange peel.}
\stopprep
\end{recipe}
\begin{recipe}{Chicken and Zucchini Burgers}
\source{\textit{Mostly Plants}}
\yield{4 burgers (4 servings, very filling)}
\acttime{45 minutes}
\ingredients
{1 lb ground chicken}
{\sfrac{3}{4} cup grated zucchini, excess moisture squeezed out}
{2 Tbsp grated yellow onion, drained of excess liquid}
{2 Tbsp ketchup}
{1 Tbsp Worcestershire sauce}
{1-3 cloves garlic, minced}
{1 Tbsp chopped basil}
{1 Tbsp chopped fresh parsley}
{1 Tbsp chopped scallion (white and light green parts)}
{Kosher salt}
{Black pepper}
{1 Tbsp vegetable/canola oil for cooking}
{4 Hamburger Buns}
{\textbf{Toppings:} Cheese, (pickled/caramelized) Onion, Lettuce (if you're nasty), Avocado, Sauce}
\stopingredients
\preparation
{Place the chicken in a large bowl. Add the zucchini, onion, ketchup, Worcestershire, garlic, basil, parsley, scallion, 1 tsp salt, and \sfrac{1}{2} tsp pepper, and mix well.}
{Lightly coat hands with oil and form into 4 patties (they will be very wet.) Place uncovered in fridge for 15 minutes to firm up.}
{In a nonstick griddle or large nonstick skillet over medium-high heat, heat the vegetable oil. Place the patties in the pan, cook for 4-5 minutes per side.}
{Top with whatever}
\stopprep
\end{recipe}
%%%%%%%%%%%%%%%%%%%%%%%%%%%%%%%%%%%%%%%%%%%%%%%%%%%%%%%%%%%%%%%%%%%%%%%%%%%%%%%%%%%%%%%%%%%%%%
\begin{recipe}{Halloumi Risotto}
\source{Adapted from \url{kitchensanctuary.com}}
\blurb{This was taken from a British website and originally called for just 2.5 oz of wine, and just 2 cups of stock. That's wrong. Lesson: never trust the British. It also called for watercress. If you really want to use that, replace 1c spinach with 1c watercress.}
\yield{4-6 servings? I think?}
\acttime{35-45 minutes}
\inacttime{in a risotto? lol}
\ingredients
{1 Tbsp olive oil}
{1 small yellow onion, finely chopped}
{1 (that's pronounced 3) cloves garlic, minced}
{1 Cup Arborio (or Carnaroli) rice}
{\sfrac{1}{2} Cup white wine}
{3 Cups (maybe more) vegetable stock}
{1 tsp vegetable/canola oil}
{4 oz Halloumi, sliced}
{1 oz grated parmesan (or romano or reggiano if you're nasty)}
{juice of \sfrac{1}{2} of a lemon}
{salt and pepper, to taste}
{2 Cups baby spinach, chopped}
{"Small handful" fresh basil, "roughly torn"}
{"Small handful" fresh parsley, "roughly torn"}
{Zest of \sfrac{1}{2} of a lemon}
{Extra lemon juice}
\stopingredients
\preparation
{Heat the vegetable stock to barely a simmer on a small burner.}
{Heat the olive oil in a large pan/dutch oven. Add the onion, cook for about 5 minutes, until onion is translucent and starting to soften.}
{Add the garlic, stir constantly and cook for another minute or so, until very fragrant.}
{Add the rice and stir constantly to toast the grains and coat with oil.}
{Add the wine and stir, wait for the wine to absorb, then add stock, a ladle at a time.}
{Once the risotto is done, heat 1 tsp vegetable oil in a small frying pan. Cook until golden brown on each side, about two to five minutes per side.}
{\textbf{Meanwhile} add the parmesan, lemon juice, salt, and pepper. Stir, then turn off the heat and add the spinach, basil, and parsley, then cover and let wilt for a couple minutes.}
{Serve.}
\stopprep
\end{recipe}
%%%%%%%%%%%%%%%%%%%%%%%%%%%%%%%%%%%%%%%%%%%%%%%%%%%%%%%
\begin{recipe}{Beet Mac and Cheese}
\blurb{This pasta is bright red. It needs some fiddling and definitely would benefit from a bit of extra texture. Will update as I try things.}
\yield{4-6 servings}
\acttime{10-15 minutes}
\inacttime{45 minutes}
\ingredients
{2 Beets, peeled and quartered/cut into 1 inch chunks}
{4 cloves garlic, minced}
{Salt and pepper, to taste}
{Olive Oil}
{1 lb Pasta (elbows, rigatoni, farfalle, and shells are good choices)}
{1 cup freshly grated parmesan}
{\sfrac{1}{2} cup freshly grated romano}
{\textbf{Optional}: \sfrac{1}{3} cup seasoned breadcrumbs and 1 Tbsp olive oil (\textbf{Not sure if amount is correct}}
{\textbf{Idea}: Some sort of roasted vegetable to mix in. Broccoli? Cauliflower?}
\stopingredients
\preparation
{Heat oven to 425\degree.}
{Drizzle beets and garlic with oil and add salt, then roast for 45 minutes, until beets are very soft.}
{While the beets are cooking, bring salted water to a boil and cook pasta until just under al dente. Drain, but \textbf{save around 2 cups of pasta water}.}
{Once the beets are out of the oven, if they're not peeled, peel them, then blend with \sfrac{1}{2}-\sfrac{3}{4} cup of pasta water until silky.}
{Add the beet puree to the pasta, and stir until combined.}
{Add the cheese, about \sfrac{1}{2} cup at a time, stirring to incorporate before adding more. Add \sfrac{1}{2}-\sfrac{3}{4} cup of the remaining pasta water and stir until smooth.}
{\textbf{Optional:} Reset the oven to 350\degree. Grease a casserole dish and add the pasta to it. Combine the olive oil and breadcrumbs and top the pasta with it. Bake for 10-15 minutes.}
\stopprep
\end{recipe}
\begin{recipe}{Fish Stew}

\blurb{To reheat this, do not microwave it. Instead, stick it in a pot over medium-low heat.}
\yield{4 servings?}
\acttime{15 minutes}
\inacttime{30 minutes}
\source{\url{https://www.simplyrecipes.com}}

\ingredients
{6 Tbsp olive oil}
{1 medium onion, chopped}
{3 (or 6) cloves garlic, minced}
{\sfrac{2}{3} c fresh parsley, chopped}
{14 oz can crushed or diced tomatoes (or 1 \sfrac{1}{2}c fresh tomato, chopped}
{2 tsp tomato paste}
{8 oz clam juice (1 c shellfish stock if you can't find that)}
{\sfrac{1}{2} c dry white wine}
{1 \sfrac{1}{2} lb firm white fish (halibut, cod, pollock, snapper, etc.}
{dried oregano, to taste (start with a pinch)}
{dried thyme, to taste (start with a pinch)}
{\sfrac{1}{8} tsp hot sauce, or more to taste (more)}
{\textbf{Optional:} It could be good to add some combination of chorizo, potatoes, kale, and carrots to this.}
{salt and pepper, to taste}
\stopingredients

\preparation
{Heat olive oil in a dutch oven over medium-high heat. Add the onion and cook for 4-6 minutes, until soft}
{Add the parsley and cook for 1 minute, stirring constantly}
{Add the garlic and cook for 1 minute or until fragrant, stirring constantly}
{Add tomatoes and tomato paste and cook for about 10 minutes, stirring occasionally}
{Add the clam juice, wine, and fish, and bring to a simmer. Cook for 3 to 5 minutes, until the fish is cooked through and flaky.}
{Season with salt, pepper, oregano, thyme, and hot sauce}
\stopprep
\end{recipe}
\end{document}
