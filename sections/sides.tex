\documentclass[../cookbook.tex]{subfiles}
\begin{document}
\begin{recipe}{Kale Caesar Salad}
\blurb{This one is mostly done off-book. Included is the recipe as written, with my annotations in
italics}
% \blurb{I know what you're thinking. A kale caesar salad? Come on, you have to have something more
% interesting than that. But it's amazing. Truly a life-changing salad}
\yield{Feeds 4}
\acttime{20 minutes, give or take}
\inacttime{1 hour (or up to 8 days)}
\source{Alton Brown: EveryDayCook}

\ingredients
{1 bunch lacinato kale, stems removed, cut into ribbons (\textit{I just use half to three quarters of a bag of regular kale from Trader Joe's and don't bother removing the stems})}
{1 bunch fresh flat-leaf parsley, stems removed, roughly chopped (\textit{This is only in there sometimes, when there's parsley in the fridge})}
{1 small shallot, frenched (\textit{This is usually in there, but just chop it as fine as you can})}
{4 Tbsp olive oil, divided}
{2 Tbsp fresh lemon juice (\textit{It's totally fine to use lemon juice from a bottle})}
{2 large garlic cloves, minced (\textit{Leave these whole (or use minced from the jar})}
{2 ounces anchovies in olive oil, finely chopped (\textit{That's a full can from Trader Joe's. And again, you don't need to chop these up})}
{3 ounces crumbled feta, divided (\textit{we usually end up using cotija because it's what's on hand})}
{2 Tbsp finely chopped preserved lemons, rinsed I assume (\textit{we have never included this, but recipe included})}
{1 c crispy chickpeas, crushed (\textit{rarely included, but they're good. Recipe included})}
{Freshly ground black pepper}
\stopingredients
\preparation
{Toss the kale, parsley, and shallot with 2 Tbsp of olive oil and let sit for 10 minutes. You can massage the kale if you want, but it's not super necessary. If it seems oily after tossing, add a bit more kale.}
{Puree the remaining oil, lemon juice, garlic, anchovies, and 1 oz of feta with a food processor, blender, or immersion blender. Adjust seasoning if neccessary, then pour over the kale.}
{If you're using the preserved lemons, add those to the salad along with the chickpeas and the remaining 2 oz of feta}
{Let sit at room temperature for 1 hour, then serve with freshly ground pepper. \textbf{Note: If you want to make ahead of time, leave the chickpeas out of it and refrigerate it, but still let it come to room temperature before serving}}
\stopprep
\end{recipe}
\begin{recipe}{Quick preserved lemons}[multi]
\blurb{Always rinse before using}
\yield{1 pint}
\acttime{10 minutes}
\inacttime{8 days}
\ingredients
{4 medium lemons, washed, scrubbed, and dried}
{\sfrac{1}{4} c coarse sea salt (or more as needed, kosher salt will also work)}
{Juice of 1 extra lemon}
\stopingredients
\vspace{-0.3cm}
\preparation
{Remove the top and tail of each lemon}
{Slice each lemon into 8 pieces and remove the seeds as you go. It's okay if some (or a lot) of juice comes out, but don't let it run off the cutting board}
{Layer the sliced lemons in a clean wide-mouthed pint jar or a 2 cup tupperware, sprinkling with salt between each layer. Don't be shy with the salt, since you're going to rinse the lemons before you use them. Pack the jar as tightly as you can.}
{Top the jar with any of the remaining juice from the board, then add the extra lemon juice. Leave about \sfrac{1}{4} inch of space at the top of the jar.}
{Refrigerate for 4 days, then flip the container over and give it another 4 days in the fridge, then sample them. The peel should be nice and soft. You can use this as is or keep it in the fridge for up to 3 months.}
{Rinse the lemons before eating. Be aware, many recipes call for the pulp to be discarded because most of the flavor is in the peel itself, not the pulp.}
\stopprep
\end{recipe}
\begin{recipe}{Crispy chickpeas}
\blurb{The methods in this recipe are heterodox. They work, but only if you follow them \underline{exactly} as written, so before you start, make sure you have the time.}
\yield{2 cups}
\acttime{10 minutes}
\inacttime{2 hours}
\ingredients
{2 15-oz cans chickpeas}
{2 Tbsp olive oil}
{1 tsp kosher salt}
{\textbf{Optional:} 1 tsp sumac, only if you are not using these for the kale salad}
{\textbf{Optional:} \sfrac{1}{4} tsp cayenne, only if you are not using these for the kale salad}
\stopingredients
\preparation
{Rinse the chickpeas in cold water. If you have a salad spinner, spin the chickpeas in the salad spinner to dry them more. Then move to a paper towel-lined sheet pan. Top with another layer of paper towels, then roll up and pat to dry them even more. You want these chickpeas to be as dry as Ben Shapiro believes women should be.}
{Remove the paper towels from the pan and toss the chickpeas with the olive oil and the salt. Put the pan (chickpeas and all) in a cold oven, set a timer for 30 minutes, THEN heat the oven to 350.}
{Once the timer goes off, turn the oven off and leave the chickpeas in there for 1 hour.}
{If you're using the chickpeas in the kale salad, you can crush them up and use them now. If you're trying for snacking chickpeas, keep going}
{Toss the warm chickpeas with sumac and cayenne, then let cool completely before storing.}
\stopprep
\end{recipe}
\end{document}
