\documentclass[../cookbook.tex]{subfiles}
\begin{document}

\begin{recipe}{Halloumi Risotto}
\source{Adapted from \url{kitchensanctuary.com}}
\blurb{This was taken from a British website and originally called for just 2.5 oz of wine, and just 2 cups of stock. That's wrong. Lesson: never trust the British. It also called for watercress. If you really want to use that, replace 1c spinach with 1c watercress.}
\yield{4-6 servings? I think?}
\acttime{35-45 minutes}
\inacttime{in a risotto? lol}
\ingredients
{1 Tbsp olive oil}
{1 small yellow onion, finely chopped}
{1 (that's pronounced 3) cloves garlic, minced}
{1 Cup Arborio (or Carnaroli) rice}
{\sfrac{1}{2} Cup white wine}
{3 Cups (maybe more) vegetable stock}
{1 tsp vegetable/canola oil}
{4 oz Halloumi, sliced}
{1 oz grated parmesan (or romano or reggiano if you're nasty)}
{juice of \sfrac{1}{2} of a lemon}
{salt and pepper, to taste}
{2 Cups baby spinach, chopped}
{"Small handful" fresh basil, "roughly torn"}
{"Small handful" fresh parsley, "roughly torn"}
{Zest of \sfrac{1}{2} of a lemon}
{Extra lemon juice}
\stopingredients
\preparation
{Heat the vegetable stock to barely a simmer on a small burner.}
{Heat the olive oil in a large pan/dutch oven. Add the onion, cook for about 5 minutes, until onion is translucent and starting to soften.}
{Add the garlic, stir constantly and cook for another minute or so, until very fragrant.}
{Add the rice and stir constantly to toast the grains and coat with oil.}
{Add the wine and stir, wait for the wine to absorb, then add stock, a ladle at a time.}
{Once the risotto is done, heat 1 tsp vegetable oil in a small frying pan. Cook until golden brown on each side, about two to five minutes per side.}
{\textbf{Meanwhile} add the parmesan, lemon juice, salt, and pepper. Stir, then turn off the heat and add the spinach, basil, and parsley, then cover and let wilt for a couple minutes.}
{Serve.}
\stopprep
\end{recipe}
%%%%%%%%%%%%%%%%%%%%%%%%%%%%%%%%%%%%%%%%%%%%%%%%%%%%%%%
\begin{recipe}{Beet Mac and Cheese}
\blurb{This pasta is bright red. It needs some fiddling and definitely would benefit from a bit of extra texture. Will update as I try things.}
\yield{4-6 servings}
\acttime{10-15 minutes}
\inacttime{45 minutes}
\ingredients
{2 Beets, peeled and quartered/cut into 1 inch chunks}
{4 cloves garlic, minced}
{Salt and pepper, to taste}
{Olive Oil}
{1 lb Pasta (elbows, rigatoni, farfalle, and shells are good choices)}
{1 cup freshly grated parmesan}
{\sfrac{1}{2} cup freshly grated romano}
{\textbf{Optional}: \sfrac{1}{3} cup seasoned breadcrumbs and 1 Tbsp olive oil (\textbf{Not sure if amount is correct}}
{\textbf{Idea}: Some sort of roasted vegetable to mix in. Broccoli? Cauliflower?}
\stopingredients
\preparation
{Heat oven to 425\degree.}
{Drizzle beets and garlic with oil and add salt, then roast for 45 minutes, until beets are very soft.}
{While the beets are cooking, bring salted water to a boil and cook pasta until just under al dente. Drain, but \textbf{save around 2 cups of pasta water}.}
{Once the beets are out of the oven, if they're not peeled, peel them, then blend with \sfrac{1}{2}-\sfrac{3}{4} cup of pasta water until silky.}
{Add the beet puree to the pasta, and stir until combined.}
{Add the cheese, about \sfrac{1}{2} cup at a time, stirring to incorporate before adding more. Add \sfrac{1}{2}-\sfrac{3}{4} cup of the remaining pasta water and stir until smooth.}
{\textbf{Optional:} Reset the oven to 350\degree. Grease a casserole dish and add the pasta to it. Combine the olive oil and breadcrumbs and top the pasta with it. Bake for 10-15 minutes.}
\stopprep
\end{recipe}
\end{document}
